\documentclass[a4paper,11pt]{article}
%\documentclass{report}

\usepackage{multicol, caption}
\usepackage{booktabs}
\usepackage{gensymb}
\usepackage[usenames,dvipsnames]{xcolor}
\usepackage{xfrac}
\usepackage{footnote}
\usepackage{textcomp}
\usepackage{dsfont}

\usepackage[latin1]{inputenc}
\usepackage[italian]{babel}
\usepackage[T1]{fontenc}
\usepackage{graphicx}
\usepackage{tabularx}
\usepackage{vmargin}
%\usepackage{multicol}
% \usepackage{subfigure}
\usepackage{caption}
% \usepackage{subcaption}
% \usepackage{upgreek}
\usepackage{rotating}
\usepackage{amssymb}
\usepackage{amsmath}
\usepackage{amsthm}
\usepackage{mathtools}
\usepackage{lscape}
\usepackage{subfig}

\newcommand{\vertiii}[1]{{\left\vert\kern-0.25ex\left\vert\kern-0.25ex\left\vert #1 
    \right\vert\kern-0.25ex\right\vert\kern-0.25ex\right\vert}}
\newcommand{\Q}{Q^{ad}}
\newcommand{\weak}{\rightharpoonup}

\newtheorem{prop}{Proposizione}
\newtheorem{teor}{Teorema}
\theoremstyle{remark}
\newtheorem{oss}{Osservazione}

%\graphicspath{{C:\Users\Ivan\Documents\Polimi\AA 2012-13\I Semestre\PACS\Condivisa_con_xubuntu-pacs\per_Fis_Tec}}
%\graphicspath{C:\Users\Ivan\Dropbox\FisicaTecnica_gruppo6\Elaborato3}
%\DeclareGraphicsExtensions{.pdf,.ps,.eps,.png,.jpeg,.mps}
%
%\bibliographystyle{plain}
%\bibliography{C:/Users/Ivan/Documents/library}

\setmarginsrb{10mm}{5mm}{10mm}{10mm}%
             {0mm}{10mm}{0mm}{10mm}

\begin{document}

\section{Caso diffusione-trasporto-reazione scalare}

Siano $I=(0,1)$, $q\in Q=H^2(I)\cap H^1_0(I)$ e
\begin{equation*}
 \Omega_q=\{(x,y)\in \mathds{R}^2\ |\ x\in I, y\in(q(x),1)\}
\end{equation*}
Consideriamo il problema di ottimizzazione di forma%, con $\Omega_q\subset \mathds{R}^2$  e $I=(0,1)$
\begin{equation}
 \min J(q,u)=\frac{1}{2}\|u-u_d\|^2_{L^2(\Omega_q)}+\frac{\alpha}{2}\|q''\|^2_{L^2(I)}+\frac{\beta}{2}\|q'\|^2_{L^2(I)}+\frac{\gamma}{2}\|q\|^2_{L^2(I)}
 \label{eq:minJ}
\end{equation}
\begin{equation}
\text{sotto il vincolo}\qquad\left\{
\begin{aligned}
 -div \left(\mu \nabla u\right) + \mathbf{b}\cdot\nabla u + div (\mathbf{c}u)+\sigma u &= f \qquad in\ \Omega_q\\
u&=g_D\qquad su\ \Gamma_D \subseteq \partial\Omega_q\\
 -\mu\partial_\mathbf{n}u + \mathbf{c}\cdot\mathbf{n}u &= g_N\qquad su\ \Gamma_N=\partial\Omega_q \backslash \Gamma_D
\end{aligned}
\right.
\label{eq:stato}
\end{equation}
La formulazione debole del problema di stato \eqref{eq:stato} \`e
\begin{equation}
	\begin{aligned}
%	Trovare\ u\in V=\{v\ |\ \exists\ v_1\in H^1_{\Gamma_D}(\Omega_q),\ \exists\ v_2\in H^1(\Omega_q): v_2|_{\Gamma_D}=g_D \text{ tali che } v=v_1+v_2\}\ tale\ che\\
\text{Trovare } u = \hat{u}+\mathcal{R}g_D \text{ con } \mathcal{R}g_D \text{ rilevamento continuo di } g_D \text{ e } \hat{u}\in V=H^1_{\Gamma_D}(\Omega_q), \text{ tale che } \\
	 a_q(\hat{u},v)=F_q(v)\qquad\forall\ v\in V\\
	dove\quad a_q(u,v)=\int_{\Omega_q}{\left[\ \mu\nabla u\cdot\nabla v + \mathbf{b}\cdot\nabla u\ v-\mathbf{c} u \cdot\nabla v+\sigma u v\ \right]}\\
	\qquad F_q(v) = \int_{\Omega_q}{f v}-\int_{\Gamma_N} {g_N v}-a_q(\mathcal{R}g_D,v)
	\end{aligned}
\label{eq:statodeb}
\end{equation}

Per evitare domini degeneri, sia $\epsilon\in (0,1)$ fissato e consideriamo solo le $q$ in
\begin{equation*}
\bar{Q}^{ad} = \{q\in Q\ |\ q(x)\leq 1-\epsilon\ \forall x\in I\}
\end{equation*}
cosicch\'e la variabile di stato $u$ sia la soluzione debole di \eqref{eq:stato} $\rightarrow u=\tilde{S}(q)$.\\
Osserviamo che la norma $\vertiii{q}= \frac{\alpha}{2}\|q''\|^2_{L^2(I)}+\frac{\beta}{2}\|q'\|^2_{L^2(I)}+\frac{\gamma}{2}\|q\|^2_{L^2(I)}$ \`e equivalente alla norma $H^2(I)$ o $H^1(I)$, purch\'e siano non nulli, rispettivamente, $\alpha$ o $\beta$, dacch\'e $Q\subset H^1_0(I)$ \  \footnote{Dimostrazione simile al Lemma 1.1 di Kiniger}. Inoltre, possiamo restringere la nostra ricerca del controllo minimizzante a $Q^{ad} = \{q\in \bar{Q}^{ad}\ |\ \vertiii{q} \leq C\}$, con, ad esempio, $C = j(q\equiv0)=J(0,\tilde{S}(0))$.

\subsection{Esistenza}
Per garantire il risultato di esistenza che riportato pi\`u avanti, possiamo utilizzare la \emph{Preposition 1.2}, che ridimostriamo nel caso in esame
\begin{prop}[Continuit\`a di $\tilde{S}$]
 Siano $q_n,q\in\Q,q_n\to q$ in $L^\infty(I)$ e $u_n=\tilde{S}(q_n)$. Allora $\exists\ \tilde{u}\in H^1_{\Gamma_D}(\hat{\Omega})$ tale che $$\tilde{u_n}\to\tilde{u}\text{ in }H^1_{\Gamma_D}(\hat{\Omega})$$
e $u=\tilde{u}|_{\Omega_q}=\tilde{S}(q)$.
\label{p:Scont}
\end{prop}
\begin{proof}
	\`E sufficiente verificare le assunzioni (A1)-(A4) di [13], pp.38 e segg.
	\begin{itemize}
		\item[A1] Uniforme continuit\`a di $a_q(u,v)\ \forall q\in Q$\\
			Basta richiedere che siano $\mu,\sigma\in L^\infty(\hat\Omega),\ \mathbf{b},\mathbf{c}\in. \left[L^\infty(\hat\Omega)\right]^2$
		\item[A2] Uniforme coercivit\`a di $a_q(u,v)\ \forall q\in Q$\\
			Basta richiedere che siano $\mu\geq\mu_0>0,\ \sigma-\frac{1}{2}div(\mathbf{b}-\mathbf{c})\geq\gamma_0\geq0$ (con $\gamma_0>0$ nel caso $\Gamma_D=\emptyset$).
		\item[A3] Simmetria di $a_q(u,v)\ \forall q\in Q$\\Non \`e verificata, ma non \`e necessaria (cfr. \emph{Remark 2.9}).
		\item[A4] Continuit\`a di $q\mapsto a_q$ (cfr. \emph{Remark 2.9})\\
			Conseguenza della uniforme continuit\`a e della dipendenza continua dell'integrale dal dominio.
	\end{itemize}
	Sotto queste ipotesi, vale il \emph{Lemma 2.12}, [13].
\end{proof}

Vale, di conseguenza il \emph{Theorem 1.3}, che riportiamo come
\begin{teor}[Esistenza]
 Il problema \eqref{eq:minJ}-\eqref{eq:stato} ammette soluzione globale.
\end{teor}
\begin{proof}
 Sia $\bar{j}=\inf_{q\in\Q}j(q)=\inf_{q\in\Q}J(q,\tilde{S(q)})$, che esiste perch\'e $\Q\neq\emptyset$ e $J(q,u)\geq0$, e sia $(q_n)_{n=1}^\infty \subseteq Q^{ad}$, con i corrispondenti $u_n=S(q_n)$, tale che $\bar{j}=\lim_{n\to\infty}j(q_n)$. Essendo $Q^{ad}$ limitato in $H^2$,\footnote{Per questa dimostrazione, serve $\alpha\neq0$, perch\'e dobbiamo stare in $H^2(I)$ per avere un risultato di immersione \textbf{compatta}} per Banach-Alaoglu negli spazi riflessivi e, poi, grazie al fatto che $H^2(I)\subset\subset C^0(I),$\footnote{Immersione di Sobolev $\Rightarrow H^2(I)\subset\subset H^1(I)\hookrightarrow C^0(I)$, perch\'e siamo in 1D} abbiamo che $\exists\ \bar{q}\in \Q$ tale che
\begin{equation*}
\begin{aligned}
 q_{n_k} \weak \bar{q}\qquad &\text{in } H^2(I)\\
 q_{n_{k_l}} \rightarrow \bar{q}\qquad &\text{in } C^0(I)
\end{aligned}
\end{equation*}
Grazie alla Proposizione \ref{p:Scont} abbiamo anche che $u_{n_{k_l}}=\tilde{S}(q_{n_{k_l}}) \rightarrow \tilde{S}(\bar{q})=\bar{u}\text{  in } V$.\\
Mostriamo ora la semicontinuit\'a inferiore debole di $j(q)=J(q,\tilde{S(q)})$: i termini nella sola $q$ costituiscono una norma ($\vertiii{q}$), dunque soddisfano la propriet\`a, mentre il primo addendo, $\frac{1}{2}\|\tilde{S(q)}-u_d\|^2_{L^2(\Omega_q)}$ \`e addirittura continuo, sempre grazie alla Proposizione \ref{p:Scont}.
Abbiamo dunque $$j(\bar{q})\leq\liminf_{l \to \infty} j(q_{n_{k_l}})=\bar{j}$$
e pertanto $(\bar{q},\bar{u})$ \`e soluzione del problema \eqref{eq:minJ}.
\end{proof}

\subsection{Formulazione alternativa (e altro)}
Scriviamo ora una formulazione alternativa di \eqref{eq:statodeb}, che fa riferimento al dominio $\Omega_0=(0,1)^2$:
\begin{equation}
	\begin{aligned}
\text{Trovare } u = \hat{u}^q+\mathcal{R}g_D \text{ con } \mathcal{R}g_D \text{ rilevamento continuo di } g_D \text{ e } \hat{u}^q=\hat{u}\circ T_q\ \in V_0=H^1_{\Gamma_D^0}(\Omega_0), \text{ tale che } \\
	 a_0(q)(\hat{u}^q,v)=F_0(q)(v)\qquad\forall\ v\in V_0\\
	dove\quad a_0(q)(u,v)=\int_{\Omega_0}{\left[\ (\mu\circ T_q)\nabla u^TA_q\nabla v +\nabla u^T DT_q^{-1}\ v \gamma_q (\mathbf{b}\circ T_q)-u \nabla v^T DT^{-1}(\mathbf{c}\circ T_q)\gamma_q+(\sigma\circ T_q) u v\gamma_q\ \right]}\\
	\qquad F_0(q)(v) = \int_{\Omega_0}{(f\circ T_q) v\gamma_q}-\textcolor{red}{\int_{\Gamma_N^0} {(g_N\circ T_q) v |DT_q \mathbf{t}|}}-a_0(q)(\mathcal{R}g_D,v)\\
	\left(\mathbf{t} = \frac{d\mathbf{X}}{ds} \text{ vettore tangente}\right)
	\end{aligned}
\label{eq:statodebT}
\end{equation}
\begin{oss}
 V. Osservazione alla formulazione alternativa di Stokes \eqref{eq:StokesdebT}
\end{oss}

\textcolor{red}{Consideriamo validi i \emph{Lemmi 1.9-1.11} di Kiniger, in quanto trattano di continuit\`a e dunque mi aspetto che siano validi, sotto ipotesi simili a quelle gi\`a poste nella Proposizione \ref{p:Scont} su $\mu,\mathbf{b},\mathbf{c},\sigma$.}\\
\textcolor{red}{REGOLARITA'?}\\
Consideriamo la stessa discretizzazione di Kiniger.\\


\subsection{Stime a priori}
Possiamo dare un risultato analogo al \emph{Corollary 3.4} di Kiniger
\begin{prop} L'operatore $S$ \`e almeno due volte continuamente Fr\'echet-differenziabile.
\end{prop}
La dimostrazione ricalca quella del citato corollario: segnaliamo la definizione di $S$ e delle sue derivate, \textcolor{red}{nel caso $g_D=0, g_N=0$, perch\'e non saprei come derivare il rilevamento e il termine di bordo} ($\cdot^q = \cdot\circ T_q$)
\begin{enumerate}
 \item $u=S(q)\in V$ \`e soluzione di \eqref{eq:statodebT}
 \item $\delta u=S'(q)(\delta q)$ \`e soluzione di 
	\begin{align*}
%	\begin{multline}
	 &a_0(q)(\delta u,v)
	+ (\nabla^Tu,\mu^qA'_{q,\delta q}\nabla v + \nabla\mu^q\cdot \mathbf{V}_{\delta q}A_q\nabla v) + \\
	&+ (v\nabla u^T,\gamma_q(DT^{-1}_q)'\delta q\mathbf{b}^q+\gamma_qDT^{-1}_q(\nabla\mathbf{b}^q\mathbf{V}_{\delta q})-DT^{-1}_q\delta q\mathbf{b}^q) + \\
	&- (u\nabla v^T,\gamma_q(DT^{-1}_q)'\delta q\mathbf{c}^q+\gamma_qDT^{-1}_q(\nabla\mathbf{c}^q\mathbf{V}_{\delta q})-DT^{-1}_q\delta q\mathbf{c}^q) + \\
	&+ (u,v(\nabla\sigma^q\cdot \mathbf{V}_{\delta q} - \sigma^q\delta q)) = \\
	&=  (\nabla f^q\cdot\mathbf{V}_{\delta q}, v\gamma_q) - (f^q,v\delta q)\\
%	\end{multline}
	\end{align*}
 \item $\delta\tau u=S''(q)(\delta q,\tau q)$ \`e soluzione di 
	\begin{align*}
	\end{align*}
	dove $\tau u = S'(q)(\tau q)$
\end{enumerate}

\section{Caso Stokes}
Consideriamo ora il problema di Stokes generalizzato
\begin{equation}
\left\{
\begin{aligned}
	\begin{aligned}
		\eta \mathbf{u} - \nu\Delta \mathbf{u} + \nabla p = \mathbf{f} \\
		div\ \mathbf{u} = 0
	\end{aligned}\qquad &in\ \Omega_q\\
	\mathbf{u} = \mathbf{g}_D\qquad &su\ \Gamma_D \subset\partial\Omega_q\\
	\nu\partial_\mathbf{n}\mathbf{u} - p \mathbf{n} = \mathbf{g}_N\qquad &su\ \Gamma_N =\partial\Omega_q\backslash\Gamma_D
\end{aligned}
\right.
\label{eq:Stokes}
\end{equation}
con formulazione debole
\begin{equation}
\begin{split}
	\text{Trovare }(\mathbf{u},p) = (\mathbf{\hat{u}}+\mathcal{R}\mathbf{g}_D,p) \text{ con } \mathcal{R}\mathbf{g}_D \text{ rilevamento continuo di } \mathbf{g}_D \text{ e } (\mathbf{\hat{u}},p)\in (V,\Pi) = ([H^1_{\Gamma_D}(\Omega_q)]^2,L^2(\Omega_q)) , \text{ tale che } \\
	\left\{
	\begin{aligned}
		a_q(\mathbf{\hat{u}},\mathbf{v}) + b_q(\mathbf{v},p) = F_q(\mathbf{v})\qquad&\forall\ \mathbf{v}\in V\\
		b_q(\mathbf{\hat{u}},\pi) = G_q(\pi)\qquad&\forall\ \pi \in \Pi\\
	\end{aligned}\right.\\
\text{dove } 
%	\begin{aligned}
		a_q(\mathbf{u},\mathbf{v})=\int_{\Omega_q}{\eta \mathbf{u}\mathbf{v}+\nu\nabla \mathbf{u}:\nabla\mathbf{v}}\\
		b_q(\mathbf{v},\pi) = -\int_{\Omega_q}{\pi\ div\ \mathbf{v}}\\
		F_q(\mathbf{v}) = \int_{\Omega_q}{ \mathbf{f}\cdot \mathbf{v}} - \int_{\Gamma_N}{\mathbf{g}_N\cdot\mathbf{v}}- a_q(\mathcal{R}\mathbf{g}_D,\mathbf{v}) \\
		G_q(\pi)=- b_q(\mathcal{R}\mathbf{g}_D,\pi)\\
%	\end{aligned}\\
\end{split}
\label{eq:Stokesdeb}
\end{equation}
\textbf{\underline{NB}} Se $\Gamma_N=\emptyset$, allora sia $\Pi=L^2_0(\Omega_q)$.

\subsection{Formulazione trasformata}
Trasferendo il problema sul dominio di riferimento $\Omega_0=(0,1)^2$ si ottiene
\begin{equation}
\begin{split}
	\text{Trovare }(\mathbf{u},p) = (\mathbf{\hat{u}}^q+\mathcal{R}\mathbf{g}_D,p^q)=(\mathbf{\hat{u}}\circ T_q+\mathcal{R}\mathbf{g}_D,p\circ T_q) \text{ con } \mathcal{R}\mathbf{g}_D \text{ rilevamento continuo di } \mathbf{g}_D \\ \text{ e } (\mathbf{\hat{u}}^q,p^q)\in (V_0,\Pi_0) = ([H^1_{\Gamma_D^0}(\Omega_0)]^2,L^2(\Omega_0)) , \text{ tale che } \\
% 	\text{Trovare }\mathbf{u} = \mathbf{\hat{u}}+\mathcal{R}\mathbf{g}_D \text{ con } \mathcal{R}\mathbf{g}_D \text{ rilevamento continuo di } \mathbf{g}_D \text{ e } \mathbf{\hat{u}}=\mathbf{\hat{u}}^q\circ T_q\text{ con }\mathbf{\hat{u}}^q\in [H^1_{\Gamma_D}(\Omega_0)]^2, \text{ tale che } \\
	\left\{
	\begin{aligned}
		a_0(q)(\mathbf{\hat{u}}^q,\mathbf{v}) + b_0(q)(\mathbf{v},p^q) = F_0(q)(\mathbf{v})\qquad&\forall\ \mathbf{v}\in V_0\\
		b_0(q)(\mathbf{\hat{u}}^q,\pi) = G_0(q)(\pi)\qquad&\forall\ \pi \in \Pi_0\\
	\end{aligned}\right.\\
\text{dove }\qquad 
%	\begin{aligned}
		a_0(q)(\mathbf{u},\mathbf{v})=\int_{\Omega_0}{\eta \mathbf{u}\cdot\mathbf{v}\gamma_q+\nu\ tr(\nabla \mathbf{u}A_q\nabla\mathbf{v}^T)}\\
		b_0(q)(\mathbf{v},\pi) = -\int_{\Omega_0}{\pi\ tr(\nabla\mathbf{v}DT_q^{-1})\gamma_q} = -\int_{\Omega_0}{\pi\ DT^{-T}:\nabla\mathbf{v}\ \gamma_q}\\
		F_0(q)(\mathbf{v}) = \int_{\Omega_0}{ \mathbf{f}\cdot \mathbf{v}\ \gamma_q} - \int_{\Gamma_N^0} {(\mathbf{g}_N\circ T_q)\cdot\mathbf{v} |DT_q \mathbf{t}|}\ - a_0(q)(\mathcal{R}\mathbf{g}_D,\mathbf{v}) \\
		G_q(\pi)= - b_0(q)(\mathcal{R}\mathbf{g}_D,\pi)
%	\end{aligned}\\
\end{split}
\label{eq:StokesdebT}
\end{equation}
\textbf{\underline{NB}} Se $\Gamma_N=\emptyset$, allora sia $\Pi_0=L^2_0(\Omega_0)$.
\begin{oss}
 $\mathbf{t}=\frac{d\mathbf{X}}{ds}$ \`e il vettore tangente. La validit\`a della formulazione riportata deriva dal fatto che nella configurazione di riferimento abbiamo i lati paralleli agli assi coordinati e lunghi 1, dunque possiamo usare come ascissa curvilinea direttamente una delle coordinate di riferimento e avere $\left|\frac{d\mathbf{X}}{ds}\right|=1$.\\
 Ora abbiamo (numerazione lati ''alla FreeFem'')
 $$DT_q(X,Y)=
  \begin{pmatrix}
   1 & 0 \\ (1-Y)q'(X) & 1-q(X)
  \end{pmatrix}
  \qquad \mathbf{t}_1=-\mathbf{t}_3=
  \begin{pmatrix}
   1 \\ 0
  \end{pmatrix}
  \qquad \mathbf{t}_2=-\mathbf{t}_4=
  \begin{pmatrix}
   0 \\ 1
  \end{pmatrix}
  $$
 e pertanto
 \begin{gather*}
   |DT_q\mathbf{t}|_{\Gamma^0_1}=|DT_q\mathbf{t}|_{\Gamma^0_3}=\sqrt{1+[(1-Y)q'(X)]^2\ } \\
   |DT_q\mathbf{t}|_{\Gamma^0_2}=|DT_q\mathbf{t}|_{\Gamma^0_4}=1-q(X)
 \end{gather*}


\end{oss}


\end{document}